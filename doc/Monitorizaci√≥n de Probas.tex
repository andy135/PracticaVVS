%%%%%%%%%%%%%%%%%%%%%%%%%%%%%%%%%%%%%%%%%%%%%%%%%%%%%%%%%%%%%%%%%%%%%%%
% This document is based on the template: Large Colored Title Article %
%                                         Version 1.1 (25/11/12)      %
%                                                                     %
% The template was downloaded from: http://www.LaTeXTemplates.com     %
%                                                                     %
% Original author:                                                    %
% Frits Wenneker (http://www.howtotex.com)                            %
%                                                                     %
% License:                                                            %
% CC BY-NC-SA 3.0 (http://creativecommons.org/licenses/by-nc-sa/3.0/) %
%                                                                     %
% Author of this version:                                             %
% Laura M. Castro (http://www.madsgroup.org/staff/laura)              %
%                                                                     %
% Original licensing terms are maintained                             %
%%%%%%%%%%%%%%%%%%%%%%%%%%%%%%%%%%%%%%%%%%%%%%%%%%%%%%%%%%%%%%%%%%%%%%%

%----------------------------------------------------------------------------------------
%	PACKAGES AND OTHER DOCUMENT CONFIGURATIONS
%----------------------------------------------------------------------------------------

\documentclass[DIV=calc,paper=a4,fontsize=11pt,onecolumn]{scrartcl}	 % A4 paper and 11pt font size

\usepackage[galician]{babel} % Galician language/hyphenation
\usepackage[utf8]{inputenc}
\usepackage[protrusion=true,expansion=true]{microtype} % Better typography
\usepackage{amsmath,amsfonts,amsthm} % Math packages
\usepackage[svgnames]{xcolor} % Enabling colors by their 'svgnames'
\usepackage[hang,small,labelfont=bf,up,textfont=it,up]{caption} % Custom captions under/above floats in tables or figures
\usepackage{booktabs} % Horizontal rules in tables
\usepackage{fix-cm}	 % Custom font sizes - used for the initial letter in the document

\usepackage{sectsty} % Enables custom section titles
\allsectionsfont{\usefont{OT1}{phv}{b}{n}} % Change the font of all section commands

\usepackage{fancyhdr} % Needed to define custom headers/footers
\pagestyle{fancy} % Enables the custom headers/footers
\usepackage{lastpage} % Used to determine the number of pages in the document (for "Page X of Total")

% Headers - all currently empty
\lhead{}
\chead{}
\rhead{}

% Footers
\lfoot{\textsc{vvs-monitorización-probas}}
\cfoot{}
\rfoot{\footnotesize Páxina \thepage\ de \pageref{LastPage}} % "Page 1 of 2"

\renewcommand{\headrulewidth}{0.0pt} % No header rule
\renewcommand{\footrulewidth}{0.4pt} % Thin footer rule

\definecolor{UDC}{RGB}{206,0,124}
\definecolor{DarkUDC}{rgb}{0.75,0.75,0.75}
\definecolor{LightUDC}{RGB}{128,128,128}

\usepackage{lettrine} % Package to accentuate the first letter of the text
\newcommand{\initial}[1]{ % Defines the command and style for the first letter
\lettrine[lines=3,lhang=0.3,nindent=0em]{
\color{UDC}
{\textsf{#1}}}{}}

%----------------------------------------------------------------------------------------
%	TITLE SECTION
%----------------------------------------------------------------------------------------

\usepackage{titling} % Allows custom title configuration

\newcommand{\HorRule}{\color{UDC} \rule{\linewidth}{1pt}} % Defines the pink horizontal rule around the title

\pretitle{\vspace{-30pt} \begin{flushleft} \HorRule \fontsize{20}{20} \usefont{OT1}{phv}{b}{n} \color{DarkUDC} \selectfont} % Horizontal rule before the title

\title{MONITORIZACIÓN DE PROBAS} % Your article title

\posttitle{\par\end{flushleft}\vskip 0.5em} % Whitespace under the title

\preauthor{\begin{flushleft}\large \lineskip 0.5em \usefont{OT1}{phv}{b}{sl} \color{DarkUDC}} % Author font configuration

\author{Título proxecto: Práctica VVS \\
        Ref. proxecto: andy135 }

\postauthor{\footnotesize \usefont{OT1}{phv}{m}{sl} \color{Black} % Configuration for the institution name
\par\end{flushleft}\HorRule} % Horizontal rule after the title

\date{\sffamily Validación e Verificación de Software} % Add a date here if you would like one to appear underneath the title block

%----------------------------------------------------------------------------------------

\usepackage{graphicx}
\usepackage{hyperref}
\hypersetup{colorlinks=true,
            allcolors=UDC}

\usepackage{array}
\usepackage{colortbl}
\usepackage{hyperref}

%----------------------------------------------------------------------------------------

\newcommand{\hint}[1]{\begin{quote}\itshape #1 \end{quote}}

%----------------------------------------------------------------------------------------

\begin{document}

\maketitle % Print the title
\thispagestyle{fancy} % Enabling the custom headers/footers for the first page 

%----------------------------------------------------------------------------------------
%	ABSTRACT
%----------------------------------------------------------------------------------------

\vspace*{1cm}

\begin{center}
\small \sffamily
  \begin{tabular}{|c|c|c|}
  \hline
  Data de aprobación & Control de versións & Observacións \\ \hline
  26/11/2015& 1.0 & Creación inicial do documento\\ \hline
  & & \\ \hline
  \end{tabular}
\end{center}

\vspace{3cm}

\begin{flushright}
	\large{\textbf{Iago Santos Domínguez}}
			
	\large{\textbf{José Andy Quintero Melo}}
    
    \large{\textbf{Elias Ferreiros Borreiros}}
\end{flushright}

\clearpage

%----------------------------------------------------------------------------------------
%	ARTICLE CONTENTS
%----------------------------------------------------------------------------------------

\section{Contexto}

\textnormal{O presente documento de monitorización de probas fai referencia a versión basica do proxecto presente en github.}

\section{Estado actual}

\begin{itemize}
	\item{Funcionalidad 01: Manejo de contenidos: Se han definido una serie de contenidos, entre los que se encuentran:}
	\begin{itemize}
		\item{Anuncio, contenido simple del que podremos obtener el título, que siempre será PUBLICIDAD, y la duración, siempre 5.}
        \item{Canción, contenido simple del que podremos obtener el título y la duración.}
		\item{Emisora es un caso de contenido del que además podremos obtener la lista de reproducción.}
	\end{itemize}	
	Persona responsable de su desarrollo: Andy Quintero Melo.
	\newline
	Persona responsable de las pruebas: Elías Ferreiro Borreiros.
	
	\item{Funcionalidad 02: Manejo de servidores: Se han definido dos tipos de servidores:}
	\begin{itemize}
		\item{Servidor Simple: En él se pueden almacenar una serie de contenidos asi como eliminarlos y buscarlos.}		
		\item{Servidor Respaldado: Realiza la misma función que un servidor simple salvo en el caso de buscar contenidos en el cual se diferencia al poseer otro servidor que le servirá de respaldo.}
	\end{itemize}
	Persona responsable de su desarrollo: Iago Santos Domínguez.	
	Persona responsable de las pruebas: Elías Ferreiro Borreiros.
\end{itemize}
\begin{center}
\small \sffamily
\begin{tabular}{|c|c|c|c|c|}
	\hline
	Funcionalidad  & Pruebas objetivo & Pruebas preparadas & \% ejecutada & \% superada 
	\\ \hline
	01 & 25 & 25 & 100 & 100
	\\ \hline
	02 & 19 & 19 & 100 & 100
	\\ \hline
\end{tabular}
\end{center}

\clearpage

\section{Registro de pruebas}

{Tipos de pruebas realizadas hasta el momento:}
\begin{itemize}
	\item{Happy testing: Durante la primera semana de testing usamos nuestro sentido común a la hora de hacer pruebas.}
	\item{Pruebas de caja negra: Al comenzar a adquirir conocimientos sobre la forma de hacer pruebas bien, usamos las siguientes técnicas:}
	\begin{itemize}
		\item{Particiones equivalentes}
		\item{Valores-frontera}
	\end{itemize}
	\item{Pruebas de caja blanca: Finalmente para ver de una manera más visual el abarque de código de las pruebas empleamos: }
	\begin{itemize}
		\item{Cobertura de instrucciones}
		\item{Cobertura de ramas}	
	\end{itemize}
    \item{Se ha descartado el uso de pruebas automatizadas debido a la simplicidad de los valores frontera y del código en general.}
\end{itemize}

\clearpage

\section{Registro de errores}

\begin{itemize}
    \item{Errores en la nomenclatura de las interfaces. \href{https://github.com/andy135/PracticaVVS/issues/2}{\#2} \href{https://github.com/andy135/PracticaVVS/issues/3}{\#3}}
    \item{Los métodos buscar de Canción y Anuncio tenían una respuesta incorrecta. \href{https://github.com/andy135/PracticaVVS/issues/5}{\#5}}
    \item{Error en la propagación de los tokens en los servidores respaldados. \href{https://github.com/andy135/PracticaVVS/issues/6}{\#6}}
    \item{Al añadir los anuncios recuperando canciones de un servidor la publicidad se añadía siempre al principio en lugar de cada tres canciones. \href{https://github.com/andy135/PracticaVVS/issues/7}{\#7}}
    \item{Al realizar las pruebas de valores frontera en los métodos de los servidores nos dimos cuenta que no contemplábamos parámetros null en la mayoría de ellos. \href{https://github.com/andy135/PracticaVVS/issues/9}{\#9}}
    \item{Al realizar las pruebas con respecto al contenido Emisora, nos dimos cuenta de que no disminuíamos la duración de la misma al eliminar algún contenido de ella. \href{https://github.com/andy135/PracticaVVS/issues/10}{\#10}}
    \item{Al realizar las pruebas de valores frontera con respecto a la duración de la canción, nos dimos cuenta de que no contemplábamos los casos en los que se especificara una duración menor o igual a 0. \href{https://github.com/andy135/PracticaVVS/issues/11}{\#11}}     
    \item{La clase SessionIdentifierGenerator debería ser estática y no lo es.\href{https://github.com/andy135/PracticaVVS/issues/13}{\#13}}
\end{itemize}

\clearpage

\section{Estadísticas}  
  \subsection{Errores encontrados diariamente y semanalmente}
	Tomando como referencia tres semanas de duración del proyecto, podríamos decir que hemos encontrado menos de un error diariamente y un poco más de dos errores semanalmente. Este valor no es muy estable ya que los días en los que nos dedicamos a la realización de las pruebas, se encontraron muchos más errores que en el resto.
  \subsection{Nivel de progreso en la ejecución de las pruebas.}
  	Toda la suite de pruebas se ejecuta sin problemas.
  \subsection{Informe de errores abiertos y cerrados por nivel de criticidad}
  	\subsubsection{Errores cerrados}
      \begin{itemize}
          \item Canciones con duración menor o igual a 0.
          \item Disminución de la duración de la Emisora al eliminar contenidos de ella.
          \item Parámetros null en métodos de servidores.
          \item Publicidad añadida siempre al principio.
          \item Clase interna no estática.
      \end{itemize}
 	\subsubsection{Errores abiertos}
    	No tenemos ningún error abierto actualmente.
  \subsection{Evaluación global del estado de calidad y estabilidad actuales}
	Podríamos afirmar con confianza que el nivel actual de calidad y de estabilidad es bueno ya que en la suite de pruebas cubrimos casi todas las posibilidades para nuestra aplicación como se puede comprobar en la Cobertura.

\clearpage

\section{Otros aspectos de interés}

Debido a las características del proyecto las herramientas usadas son JUnit (automatización de pruebas), Cobertura (Validar la calidad de nuestras pruebas), FindBugs (pruebas estructurales), Checkstyle (pruebas estructurales) y PMD (pruebas estructurales).
Hemos decidido no usar herramientas de generación de datos debido a la simplicidad de los valores frontera de nuestros métodos.

\end{document}
